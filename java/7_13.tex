\documentclass{jsarticle}
\usepackage[dvipdfmx]{graphicx}
\title{数値解析}
\date{\today}

\usepackage[height=26cm,width=16cm]{geometry}

\begin{document}

\maketitle

\section{数値微分}
点列データから数値を計算する。
\subsection{例}
${(x_0,f_0)(x_1,f_1)(x_2,f_2)}を2次ラグランジュ多項式で補間する
\begin{equation}
  P(x)=f_0\frac((x-x_1)(x-x_2))((x_0-x_1)(x_0-x_2))+f_1\frac((x-x_0)(x-x_2))(x_1-x_0)(x_1-x_2))+f_2\frac((x-x_0)(x-x_1))((x_2-x_0)(x_2-x_1))
  P(x)'=f_0\frac(2x-x_1-x_2)((x_0-x_1)(x_0-x_2))+f_1\frac(2x-x_0-x_2)((x_1-x_0)(x_1-x_2))+f_2
\end{equation}

\subsection{関数f(x)が与えられて、ある点x_0での微係数を計算する場合}
\begin{equation}
  f(x_0+h)=f(x_0)+hf'(x_0)+\frac(h^2)(2!)f''(x_0)+\frac(h^3)(3!)f^{3}(x_0)・・・
  f(x_0-h)=f(x_0)-hf'(x_0)+\frac(h^2)(2!)f''(x_0)-\frac(h^3)(3!)f^{3}(x_0)・・・
  f(x_0+h)-f(x_0-h)=2hf'(x_0)+0(h^3)
  f'(x_0)=\frac(f(x_0+h)-f(x_0-h))(2h)
  f(x_0+h)-f(x_0-h)=2f(x_0)h^2f''(x_0)+0(h^2)
  f''(x_0)=\frac(f(x_0+h)-2f(x_0)+f(x_0-h))(h^2))
\end{equation}
\section{リチャードソンの外挿}
\begin{equation}
  f'(x_0)=\frac(f(x_0+h)-f(x_0-h))(2h)-\frac(1)(6)h^2f^{3}(x_0)-\frac(1)(120)h^4f^{5}(x_0)・・・(1)
  hをrhに置き換える
  f'(x_0)=\frac(f(x_0+rh)-f(x_0-rh))(2rh)-\frac(1)(6)(rh)^2f^{3}(x_0)-\frac(1)(120)(rh)^4f^{5}(x_0)・・・(2)
  (2)-r^2*(1)
\end{equation}
\section{数積分}
区間${[a,b]}$をいくつかの区間二分割し、点${x_1}$に対して補間関数を積分することで積分値を求める(ニュートン・コーツ系の積分公式)
\subsection{台形則}
${x_0,x_1)}$を線形ラグランジュ補間多項式で積分する
\begin{equation}
  \int_{x_0}_{x_1}f(x)dx≒int_{x_0}_{x_1}p(x)
\end{equation}
\subsection{シンプソン則}
2次のラグランジュ多項式。2つの区間の面積を計算するのに3点いる。\\


\end{document}
