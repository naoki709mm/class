\documentclass{jsarticle}
\usepackage[dvipdfmx]{graphicx}

\title{電磁誘導}
\date{\today}

\usepackage{bm}
\usepackage[height=26cm,width=16cm]{geometry}

\begin{document}

\maketitle

\section{}
\subsection{}
	電磁場が時間変化する場合を考える。ファラデーは電流が磁場を作るなら、逆に、磁場が電流を作るはずだと考えた。そして、2つのコイルを近くに並べ、一方に、電流を流した。しかし、電流を流し続けても何も起こらなかった。
	
	片方の回路に検電器、片方にスイッチ付きの回路を用意し、後者の回路をオンオフするときに磁場の時間変化が発生。

\begin{itemize}
	\item 回路に生じる起電力$\phi$ は回路を貫く磁束の時間変化、$\frac{dt}{d\phi}$に比例する。
	
	\item 一様な磁束密度{\bf B}が、面積Sの平面を貫く時
	 	\begin{equation}
		 	\Phi=BS
		\end{equation}
										
		一般
		\begin{equation}
			\Phi=\int_S \bm{B} (\bm{r}) \bm{n} (r)dS
		\end{equation}
	
		F.Naunmanが数式化
		\begin{equation}
			\phi=-{\it k} \frac(dt)(d\Phi)
		\end{equation}
	
		単位
		\begin{equation}
			[\Phi]=[wb] :ウェーバー[\phi]=[V] :ボルト
		\end{equation}
	
	\item レンツの法則
		回路Cの緑にした、曲面Swo貫く磁束$\Phi$ は曲面によらず一定。\\
		任意の閉じた回路Cについて
		\begin{equation}
			\phi=\int_C {\it E(r, t)t(r)} ds = -\frac{dt}{d\Phi(t)}
		\end{equation}
		\begin{equation}
			\int_C \bm{E}(\bm{r},t)*t(r) ds = \int_S \Delta \times \bm{E}(\bm{r},t) \bm{n}(r) dS
		\end{equation}
		\begin{equation}
			-\frac{dt}{d\Phi(t)}=\frac{dt}{d}\int_S \bm{B}(\bm(r),t)*\bm{n}(\bm{r}) dS 
			             =-\int_S \frac{\partial t}{\partial \bm{B}(\bm{r},t)}*\bm{n}(\bm{r})
		\end{equation}
		\begin{equation}
			\Delta \times \bm{B}(\bm{r},t) = -\frac{\partial t}{\partial \bm{B}(\bm(r),t)}
		\end{equation}
		微分系の電磁誘導の法則\\
		これは回路がなくても成り立つ。空間経路を考えればよい。
\end{itemize}		
\subsection{}
	運動する回路ないの起電力$\bm{B}(\bm{r})$は時間変化しないとする。この$\bm{B}$中を回路Cを速度$\bm{v}$で動かしたとする
\section{}
	自己インダクタンスはやらない
\end{document}	