\documentclass{jsarticle} 

\usepackage{enumerate}

\title{複素関数論 レポート}
\author{ 1422020156 奥屋 直己}

\usepackage[height=26cm,width=16cm]{geometry}

\begin{document}

\maketitle

\section{次の式の複素共役な式を作りなさい}
	\begin{enumerate}[(1)]
		\item $\frac{i}{2+3i}$\\
			分母分子にに(2-3i)をかけると\\
			\begin{eqnarray*}
				\frac{i}{2+3i}=\frac{i(2-3i)}{(2+3i)(2-3i)}=\frac{2i-3i^2}{4+9i^2}=\frac{3+2i}{4-9}=-\frac{3}{5}-\frac{2}{5}i
			\end{eqnarray*}
			よって、複素共役数は
			\[
				u^*=-\frac{3}{5}+\frac{2}{5}i
			\]
		\item $e^{(2+5i)π}$ \\
			式変形を行うと
			\[
				e^{(2+5i)\pi}=e^{(2\pi+5i\pi)}=e^{2\pi}e^{5i\pi}
			\]
			オイラーの公式より
			\[
				e^{2\pi}e^{5i\pi}=e^{2\pi}(\cos5\pi+i\sin5\pi)=e^{2\pi}(1+0)=e^{2\pi}
			\]
			よって虚部が0になるため、複素共役は
			\[
				u^*=e^{2\pi}
			\]
		\item $(\sqrt{3}-i)^\frac{3}{2}$\\
			$(\sqrt(3)-i)^3$を展開すると
			\[
				(\sqrt{3}-i)^3=(3-2\sqrt{3}i-1)(\sqrt{3}-i)=(2-2\sqrt{3})(\sqrt{3}-i)=2\sqrt{3}-2i-6i-2\sqrt{3}=-8i
			\]
			よって
			\[
				(\sqrt{3}-i)^\frac{3}{2}=\sqrt{-8i}=2\sqrt{2}i^2=-2\sqrt{2}
			\]
			よって虚部が0になったので、複素共役は
			\[
				u^*=-2\sqrt{2}
			\]
			
		\item $\log{(-2+2i)}$ \\
			
	\end{enumerate}
\end{document}












